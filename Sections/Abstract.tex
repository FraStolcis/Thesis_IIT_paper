%----------------------------------------------------------------------------------------
%	ABSTRACT PAGE
%----------------------------------------------------------------------------------------

\begin{abstract}
\addchaptertocentry{\abstractname} % Add the abstract to the table of contents
This thesis proposes a novel approach for enhancing the teleoperation of robotic arms through shared control mechanisms, integrating advanced machine learning techniques and classical control methods. The primary objective is to develop a robust system capable of autonomously guiding the robotic arm towards predicted objects while maintaining operator oversight. The framework employs a Long Short-Term Memory (LSTM) classification model to predict the location and nature of objects within the robot's workspace. This predictive capability enables the system to anticipate the operator's intentions and adaptively plan trajectories. Furthermore, artificial potential fields are utilized to generate guidance commands that assist the operator in maneuvering the robotic arm towards the predicted objects. By combining the predictive power of LSTM with the reactive nature of potential fields, the system achieves a seamless fusion of human expertise and autonomous control, ensuring both efficiency and safety in teleoperation tasks. The proposed approach is implemented and validated through comprehensive simulations and real-world experiments using a teleoperated robotic arm platform. Performance evaluations demonstrate the effectiveness and reliability of the shared control system, showcasing improved object manipulation capabilities and reduced cognitive workload for the operator. Additionally, the system's adaptability to varying environmental conditions and object dynamics is examined, highlighting its potential for deployment in diverse teleoperation scenarios. Overall, this thesis contributes to the advancement of teleoperated robotic systems by introducing a novel framework that seamlessly integrates machine learning and classical control techniques. The proposed shared control approach offers enhanced capabilities for object manipulation tasks, paving the way for more efficient and intuitive human-robot collaboration in various domains, including manufacturing, healthcare, and disaster response.
\end{abstract}